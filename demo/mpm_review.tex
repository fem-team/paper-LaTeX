%% LyX 2.0.2 created this file.  For more info, see http://www.lyx.org/.
%% Do not edit unless you really know what you are doing.
\documentclass{ctexart}
\usepackage{amsmath}
\usepackage{fontspec}
\usepackage{geometry}
\geometry{verbose,tmargin=2cm,bmargin=2cm,lmargin=2cm,rmargin=2cm}
\usepackage{color}
\usepackage{graphicx}
\usepackage{esint}
\usepackage[unicode=true,pdfusetitle,
 bookmarks=true,bookmarksnumbered=false,bookmarksopen=false,
 breaklinks=false,pdfborder={0 0 1},backref=false,colorlinks=false]
 {hyperref}

\makeatletter

%%%%%%%%%%%%%%%%%%%%%%%%%%%%%% LyX specific LaTeX commands.
%% Because html converters don't know tabularnewline
\providecommand{\tabularnewline}{\\}
%% A simple dot to overcome graphicx limitations
\newcommand{\lyxdot}{.}


%%%%%%%%%%%%%%%%%%%%%%%%%%%%%% Textclass specific LaTeX commands.
\newcommand{\lyxaddress}[1]{
\par {\raggedright #1
\vspace{1.4em}
\noindent\par}
}

%%%%%%%%%%%%%%%%%%%%%%%%%%%%%% User specified LaTeX commands.
%!TEX TS-program = xelatex

\usepackage[super,square,comma,sort&compress]{natbib}

\makeatother

\usepackage{xunicode}
\begin{document}

\title{有限元课程论文}


\author{康金梁,郑蕴哲,司马锲,吉首瑞}

\maketitle

\lyxaddress{\begin{center}
(清华大学航天航空学院,北京100084)
\par\end{center}}
\begin{abstract}


关键字:无网格法,物质点法,冲击侵彻,爆炸,特大变形

\end{abstract}

\section{引言}



\section{STAPpp程序设计与实现}


\subsection{前处理}

\subsection{Bar单元}

\subsection{4Q单元}

\subsection{3T单元}

\subsection{8H单元}

\subsection{Beam单元}

\subsection{Plate单元}

\subsection{Shell单元}

\subsection{IEM单元}

\subsection{若干“巧凑”边点元}

\subsection{求解器优化}

\subsection{后处理}




\begin{figure}[tbh]
\centering\includegraphics{\lyxdot \lyxdot /figures/MPMFlowChart}

\caption{物质点法与显式有限元计算流程比较}


\label{MPMvsFEM}
\end{figure}






\section{桥梁设计、优化与评估}




\section{结论}



\bibliographystyle{unsrt}
\bibliography{../ref/refs1-utf8}

\end{document}
