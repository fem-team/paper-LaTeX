%% LyX 2.0.2 created this file.  For more info, see http://www.lyx.org/.
%% Do not edit unless you really know what you are doing.
\documentclass{ctexart}
\usepackage{amsmath}
\usepackage{fontspec}
\usepackage{geometry}
\geometry{verbose,tmargin=2cm,bmargin=2cm,lmargin=2cm,rmargin=2cm}
\usepackage{color}
\usepackage{graphicx}
\usepackage{esint}
\usepackage{tikz}
\usepackage{pgfplots}
\usepackage{float}
\usepackage{listings}
\usepackage{graphicx}
\usepackage{amsmath}
\usepackage{subfigure}
\usepackage[unicode=true,pdfusetitle,
 bookmarks=true,bookmarksnumbered=false,bookmarksopen=false,
 breaklinks=false,pdfborder={0 0 1},backref=false,colorlinks=false]
 {hyperref}

\makeatletter

%%%%%%%%%%%%%%%%%%%%%%%%%%%%%% LyX specific LaTeX commands.
%% Because html converters don't know tabularnewline
\providecommand{\tabularnewline}{\\}
%% A simple dot to overcome graphicx limitations
\newcommand{\lyxdot}{.}


%%%%%%%%%%%%%%%%%%%%%%%%%%%%%% Textclass specific LaTeX commands.
\newcommand{\lyxaddress}[1]{
\par {\raggedright #1
\vspace{1.4em}
\noindent\par}
}

%%%%%%%%%%%%%%%%%%%%%%%%%%%%%% User specified LaTeX commands.
%!TEX TS-program = xelatex

\usepackage[super,square,comma,sort&compress]{natbib}

\makeatother

\usepackage{xunicode}
\begin{document}

\title{有限元课程论文}


\author{康金梁,郑蕴哲,司马锲,吉首瑞}

\maketitle

\lyxaddress{\begin{center}
(清华大学航天航空学院,北京100084)
\par\end{center}}
\begin{abstract}


关键字:无网格法,物质点法,冲击侵彻,爆炸,特大变形

\end{abstract}

\section{引言}



\section{STAPpp程序设计与实现}


\subsection{前处理}

\subsection{Bar单元}

\subsection{4Q单元}

\subsection{3T单元}

\subsubsection{3T单元求解过程}

3T-1求形函数N

引入面积坐标对求3T单元很有效
\begin{figure}[H] 
	\centering 
	\includegraphics[width=0.3\textwidth,height=0.12\textheight]{1} 
	\caption{面积坐标} 
	\label{} 
\end{figure}

通过面积坐标的公式:
\begin{figure}[H] 
	\centering 
	\includegraphics[width=0.6\textwidth,height=0.12\textheight]{2} 
	\caption{变换公式} 
	\label{} 
\end{figure}
并且N=ξ
得到B函数如下:
\begin{figure}[H] 
	\centering 
	\includegraphics[width=0.6\textwidth,height=0.12\textheight]{3} 
	\caption{B矩阵} 
	\label{} 
\end{figure}
因此,通过公式可得到单元K矩阵:
\begin{figure}[H] 
	\centering 
	\includegraphics[width=0.3\textwidth,height=0.04\textheight]{4} 
	\caption{单元K矩阵} 
	\label{} 
\end{figure}

3T-2外力矩阵

集中力仍然不需要修改(未进行坐标变换)
体作用力公式如下:
\begin{figure}[H] 
	\centering 
	\includegraphics[width=0.8\textwidth,height=0.19\textheight]{5} 
	\caption{体作用力公式} 
	\label{} 
\end{figure}
以及面作用力如下:(通过一点高斯积分获得)
\begin{figure}[H] 
	\centering 
	\includegraphics[width=0.8\textwidth,height=0.19\textheight]{6} 
	\caption{面作用力公式} 
	\label{} 
\end{figure}
如此获得了单元刚度阵与外力阵


3T-3组装

按照事先划好的网格进行组装(LM阵已知)
\begin{figure}[H] 
	\centering 
	\includegraphics[width=0.4\textwidth,height=0.06\textheight]{7} 
	\caption{组装公式} 
	\label{} 
\end{figure}


3T-4求解

最后通过矩阵的直接求解法(LDLT分解法)求解。
反作用力由以下公式求出:
\begin{figure}[H] 
	\centering 
	\includegraphics[width=0.4\textwidth,height=0.06\textheight]{8} 
	\caption{反作用力公式} 
	\label{} 
\end{figure}

位移可直接求出。
通过位移,按照以下公式可得应变、应力:
\begin{figure}[H] 
	\centering 
	\includegraphics[width=0.8\textwidth,height=0.10\textheight]{9} 
	\caption{应变应力公式} 
	\label{} 
\end{figure}

按照以上思路完成3T cpp程序的写作。
\subsubsection{Pacth test 与收敛性分析}

Patch Test部分:

通过单项给定的线性位移场进行:
图形如下:(材料参数E=1000,v=0.3)
\begin{figure}[H] 
	\centering 
	\includegraphics[width=0.4\textwidth,height=0.24\textheight]{10} 
	\caption{测试图形} 
	\label{} 
\end{figure}

给定位移场为(单向拉伸):
\begin{figure}[H] 
	\centering 
	\includegraphics[width=0.4\textwidth,height=0.08\textheight]{11} 
	\caption{线性位移} 
	\label{} 
\end{figure}
输入参数设定顺时针,且固定左下点与左上x方向):


获得的位移结果为:
\begin{figure}[H] 
	\centering 
	\includegraphics[width=0.8\textwidth,height=0.10\textheight]{12} 
	\caption{位移结果} 
	\label{} 
\end{figure}
可见误差项为10-17量级,属于机器误差范围,因此模型通过了Patch test.
故本程序是收敛的。

收敛率分析部分

对左下角固定的正方形材料进行拉伸:(E=1000,v=0.03)
\begin{figure}[H] 
	\centering 
	\includegraphics[width=0.8\textwidth,height=0.18\textheight]{13} 
	\caption{加密图形} 
	\label{} 
\end{figure}
采取细分网格如图,单元给定线性位移场
u=0.002x
单向拉伸的正方形单元的收敛率分析:
精确解的:普通单元的常数应变为:

Ex=0.02

Ey=Ez=-0.006

\begin{figure}[H] 
	\centering 
	\includegraphics[width=0.4\textwidth,height=0.05\textheight]{19} 
	\caption{计算能量范数下误差公式} 

\end{figure}

a.2单元

\begin{figure}[H] 
	\centering 
	\includegraphics[width=0.8\textwidth,height=0.16\textheight]{14} 
	\caption{2单元应力数据} 
	\label{} 
\end{figure}

得到应变为:

element Ex	Ey	Ez

[1]	0.01944	-0.00583	-0.00559

[2]	0.02056	-0.00559	-0.00559

ABS(e)=0.0328


b.4单元
\begin{figure}[H] 
	\centering 
	\includegraphics[width=0.8\textwidth,height=0.17\textheight]{15} 
	\caption{4单元应力数据} 
	\label{} 
\end{figure}

element Ex	Ey	Ez

[1]	0.01924	-0.00577	0

[2]	0.02	-0.00217	0.00762

[4]	0.02076	0.001415	0

[4]	0.02	-0.00217	-0.00762

ABS(e)=0.0162

c.8单元

\begin{figure}[H] 
	\centering 
	\includegraphics[width=0.8\textwidth,height=0.22\textheight]{16} 
	\caption{8单元应力数据} 
	\label{} 
\end{figure}

element Ex	Ey	Ez

[1]	V	-0.00548	0

[2]	0.0183	-0.00548	0

[3]	0.0217	-0.000758	0.000349

[4]	0.0218	-0.00111	-0.00152

[5]	0.0191	0.00152	-0.00378

[6]	0.0191	0.00152	0.00378

[7]	0.0218	-0.00111	0.00152

[8]	0.0217	-0.000758	-0.000349

ABS(e)=0.0090

对结果进行线性拟合(e的对数)
得到:

P=1.11

故最后该3T单元的收敛率为1.11,大于1因此收敛(通过Pacth Test已经收敛。)


\subsubsection{本人所承担的小组任务}
1. 3t单元的整合与调试

2. 求解器优化,做了相关调研和前期准备,后续由于电脑故障无法进行交予组员

3. 桥梁参数计算、拉索分散模型和后续文档


\subsection{8H单元}

\subsection{Beam单元}

\subsection{Plate单元}

\subsection{Shell单元}

\subsection{IEM单元}

\subsection{若干“巧凑”边点元}

\subsection{求解器优化}

\subsection{后处理}









\section{桥梁设计、优化与评估}




\section{结论}



\bibliographystyle{unsrt}
\bibliography{../ref/refs1-utf8}

\end{document}
